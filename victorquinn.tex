%!TEX TS-program = xelatex
\documentclass[]{friggeri-cv}
\usepackage{afterpage}
\usepackage{hyperref}
\usepackage{color}
\usepackage{xcolor}
\usepackage{smartdiagram}
\usepackage{fontspec}
% if you want to add fontawesome package
% you need to compile the tex file with LuaLaTeX
% References:
%   http://texdoc.net/texmf-dist/doc/latex/fontawesome/fontawesome.pdf
%   https://www.ctan.org/tex-archive/fonts/fontawesome?lang=en
%\usepackage{fontawesome}
\usepackage{metalogo}
\usepackage{dtklogos}
\usepackage[utf8]{inputenc}
\usepackage{tikz}
\usetikzlibrary{mindmap,shadows}
\hypersetup{
    pdftitle={},
    pdfauthor={},
    pdfsubject={},
    pdfkeywords={},
    colorlinks=false,           % no lik border color
    allbordercolors=white       % white border color for all
}
\smartdiagramset{
    bubble center node font = \footnotesize,
    bubble node font = \footnotesize,
    % specifies the minimum size of the bubble center node
    bubble center node size = 0.5cm,
    %  specifies the minimum size of the bubbles
    bubble node size = 0.5cm,
    % specifies which is the distance among the bubble center node and the other bubbles
    distance center/other bubbles = 0.3cm,
    % sets the distance from the text to the border of the bubble center node
    distance text center bubble = 0.5cm,
    % set center bubble color
    bubble center node color = pblue,
    % define the list of colors usable in the diagram
    set color list = {lightgray, materialcyan, orange, green, materialorange, materialteal, materialamber, materialindigo, materialgreen, materiallime},
    % sets the opacity at which the bubbles are shown
    bubble fill opacity = 0.6,
    % sets the opacity at which the bubble text is shown
    bubble text opacity = 0.5,
}

\addbibresource{bibliography.bib}
\RequirePackage{xcolor}
\definecolor{pblue}{HTML}{0395DE}

\begin{document}
\header{Victor}{Quinn}
      {Software Engineer and Technical Leader}
      
% Fake text to add separator      
\fcolorbox{white}{gray}{\parbox{\dimexpr\textwidth-2\fboxsep-2\fboxrule}{%
.....
}}

% In the aside, each new line forces a line break
\begin{aside}
  \includegraphics[scale=0.55]{img/HeadshotCircle.png}
  \section{Address}
    2727 S Quincy St
    Apt 1215
    Arlington, VA 22206
    ~
  \section{Tel}
    (413) 687-2928
    ~
  \section{Mail}
    \href{mailto:mail@victorquinn.com}{{mail@victorquinn.com}}
    ~
  \section{Web \& Git}
    \href{https://www.victorquinn.com}{www.victorquinn.com}
    \href{https://github.com/victorquinn}{github.com/victorquinn}
    ~
  % use  \hspace{} or \vspace{} to change bubble size, if needed
  \section{Programming}
    \smartdiagram[bubble diagram]{
        \textbf{Node},
        \textbf{JavaScript},
        \textbf{HTML}\\\textbf{CSS},
        \textbf{React}\\\textbf{Angular},
        \textbf{Other\vspace{1mm}},
        \textbf{Go\vspace{1mm}},
        \textbf{Python},
        \textbf{iOS}\\\textbf{Android},
        \textbf{Bash}
    }
    ~
  \section{Personal Skills}
    \smartdiagram[bubble diagram]{
        \textbf{Team}\\\textbf{Player},
        \textbf{Initiative},
        \textbf{Curiosity},
        \textbf{Problem}\\\textbf{Solving},
        \textbf{Leadership},
        \textbf{Organize}
    }
    ~
\end{aside}
~
\section{Experience}
\begin{entrylist}
  \entry
    {09/13 - Now}
    {VP of Engineering}
    {SocialRadar}
    {Leading the engineering team at SocialRadar, a startup specializing in location technologies. Architecting systems and ensuring applications are built to scale. Managing engineers by building out the team, working to keep morale and retention up, maintaining a positive engineering culture, providing technical guidance, acting as a mentor and advocate, and leading by example. Polyglot coder, committing code to anything needing attention on the front end, back end, or ops. Led team through multiple pivots.\\}
  \entry
    {03/12 - 09/13}
    {Senior Software Engineer}
    {NGP VAN}
    {Led an engineering team to revamp the process by which the company took and processed contributions for political campaigns. Spearheaded a rewrite of the core API for taking monetary contributions and the dynamic client-side form generation. This resulted in forms that were more extensible, better looking, and auto-filling which led to significantly increased conversion rates.\\}
    \entry
    {08/08 - 03/12}
    {Web Programmer}
    {Amherst College}
    {Collaborated with a team to create custom tools to empower faculty and staff to create rich education content. Primarily focused on the full stack of media uploads, from UI to storage, my work resulted in making it easier for educators to create content, students to upload files, and educators to review them.\\}
    \entry
    {07/05 - 08/08}
    {Senior Developer}
    {University of Massachusetts Amherst}
    {Specialized in software design and programming for internal processes including web development, custom Windows application development, and other various scripting. Each project I developed helped automate some processes and liberate others from antiquated systems such as moving from an emailed MS Access file to a true database application.}
\end{entrylist}
\\
\section{Education}
\begin{entrylist}
  \entry
    {2007 - 2011}
    {Juris Doctor}
    {Western New England University School of Law}
    {Attended part-time at night while working full-time as a programmer during the day. Elected President of the IP Law Association. Maintained a course concentration in Intellectual Property Law including an Independent Study in Patent Litigation. Awarded two Excellence for the Future Awards for the Highest Achievement in Analytical Methods for Lawyers and Trademark Law.\\}
  \entry
    {2001 - 2006}
    {Bachelor's Degree in Computer Science}
    {University of Massachusetts Amherst}
    {Member of the CS Talent Advancement Program and a Computer Science program Mentor. Advised and aided younger CS majors.\\}
  \entry
    {2001 - 2006}
    {Bachelor's Degree in Physics}
    {University of Massachusetts Amherst}
    {Completed alongside Computer Science degree. Focused on optics and theoretical physics.}
\end{entrylist}

\newpage

\begin{aside}
~
~
~
  \section{Places Lived}
    \includegraphics[scale=0.25]{img/PlacesLived.png}
    ~
  \section{Frontend Framework Preference}
    \textbf{React}\includegraphics[scale=0.40]{img/5stars.png}
    \textbf{Angular}\includegraphics[scale=0.40]{img/4stars.png}
    \textbf{Ember}\includegraphics[scale=0.40]{img/3stars.png}
    ~
  \section{My Bike}
    \includegraphics[scale=0.05]{img/HarleyCropped.jpg}
    {Yes my license plate actually is \textit{Emacs}}
    ~
   \section{Backend Language Preference}
    \textbf{Node.js}\includegraphics[scale=0.40]{img/5stars.png}
    \textbf{Go}\includegraphics[scale=0.40]{img/4stars.png}
    \textbf{Rust}\includegraphics[scale=0.40]{img/4stars.png}
    \textbf{Python}\includegraphics[scale=0.40]{img/3stars.png}
    \textbf{Ruby}\includegraphics[scale=0.40]{img/2stars.png}
    ~
\end{aside}

\section{Publications}
\textbf{\href{http://gettingstartedwithtmux.com/}{Getting Started with tmux}}\\
{I authored this technical book to help developers make use of one of my favorite command line tools. Published by Packt Publishing, this easy and approachable getting started guide is a great point of entry for developers looking to maximize their terminal usage. However, it also has enough depth that it is helpful for more adept tmux users as well.}
\\
\section{Open Source}
\begin{entrylist}
  \entry
  {06/2013}
  {\href{http://chancejs.com}{Chance}}
  {Creator}
  {Chance is a random generator helper for JavaScript. I wrote it out of frustration for lack of libraries like it. Now approaching 2000 stars on Github, it was spun off to its own organization and has been forked over 160 times.}
  \entry
  {08/2013}
  {\href{http://dynastyjs.com}{Dynasty}}
  {Creator}
  {Promise-based DynamoDB client with a focus on simplicity. I wrote this after being unsatisfied with the core AWS library due to its use of traditional Node callbacks and confusing API.}
  \entry
  {11/2013}
  {\href{http://batch-request.socialradar.com}{Batch Request}}
  {Creator}
  {Lightweight connect/express middleware for Node.js which allows clients to send multiple requests to a server running Node.js in batch. Clients can send a single request that represents many, have them all run, then return a single response.}
  \entry
  {06/2014}
  {\href{http://memcache-plus.com/}{Memcache Plus}}
  {Creator}
  {A better Memcache client for Node.js featuring Promises, baked in support for Elasticache Auto Discovery, command buffering, per-key compression, and more. I wrote this as I was unsatisfied with the other existing Memcache clients.}
\end{entrylist}

\section{Other Achievements}
\begin{entrylist}
    \entry
    {01/2012}
    {Registered Patent Agent}
    {}
    {US Patent and Trademark Office Agent \#69321}
    \entry
    {01/2000}
    {Eagle Scout}
    {}
    {The highest rank in the Boy Scouts of America, achieved by only 4\% of scouts.}
\end{entrylist}
\\
\section{About Me}
I type with the Dvorak keyboard layout on a Kinesis Advantage keyboard while using Emacs as my primary tool to write code. I remap Control to Caps Lock to help me fly and I'm an efficiency junkie who is always looking for ways to improve my workflow. I'm passionate about a great many things in the world of computing, but a few of my favorites are databases, architecture, and clean API design. I currently live in the Washington, D.C. metro area with my wife and Great Dane where I enjoy writing fiction, brewing my own beer, and riding my Harley.
\\
\begin{flushleft}
\emph{August 7, 2016}
\end{flushleft}
\begin{flushright}
\emph{Victor Quinn}
\end{flushright}

\end{document}
